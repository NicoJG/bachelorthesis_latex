% This example is meant to be compiled with lualatex or xelatex
% The theme itself also supports pdflatex
\PassOptionsToPackage{unicode}{hyperref}
\documentclass[aspectratio=1610, 9pt]{beamer}

% Load packages you need here
\usepackage[ngerman, english]{babel}

\usepackage[autostyle]{csquotes}

\usepackage{amsmath}
\usepackage{amssymb}
\usepackage{mathtools}
%\input{lhcb-symbols-def.tex}

% Enable Unicode-Math and follow the ISO-Standards for typesetting math
\usepackage{unicode-math}

\usepackage[
  separate-uncertainty=true,
  per-mode=symbol-or-fraction,
]{siunitx}
% automatically choose the right locale
\addto\extrasngerman{\sisetup{locale = DE}}
\addto\extrasenglish{\sisetup{locale = UK}}

\usepackage{graphicx}
%\usepackage{enumitem}

\usepackage{hyperref}
\usepackage{bookmark}

% load the theme after all packages

\usetheme[
  showtotalframes, % show total number of frames in the footline
]{tudo}

% Put settings here, like
\unimathsetup{
  math-style=ISO,
  bold-style=ISO,
  sans-style=italic,
  nabla=upright,
  partial=upright,
  warnings-off={mathtools-colon,mathtools-overbracket}, % suppress some unnecessary warnings
}

\title{Development of a multivariate algorithm for the classification of B~mesons\\at the LHCb experiment}
\author[N.~Guth]{Nico Guth}
\institute[AG Albrecht]{Arbeitsgruppe Albrecht \\ Fakultät Physik}
\date{Bachelor talk, 20.07.2022}
%\subtitle{Bachelorthesis}
%\titlegraphic{\includegraphics[width=0.7\textwidth]{images/tudo-title-2.jpg}}

\begin{document}

\maketitle

\begin{frame}{Overview}
  \textbf{Goal of my thesis:} Use machine learning algorithms to distinguish between $B^0_d$ and $B^0_s$~mesons based on the kinematics of the associated event in proton-proton collisions at the LHCb detector. 

  \medskip
  \textbf{Structure of this talk:}
  \begin{itemize}
    \item Motivation
    \item Theoretical and experimental background
      \begin{itemize}
        \item The Standard Model
        \item $B$ meson production in proton-proton collisions
        \item The LHCb detector
        \item Classification algorithms
      \end{itemize}
    \item Development of the $B$ meson classifier
      \begin{itemize}
        \item Identification of same side tracks using a BDT
        \item Classification of the $B$ meson using a DeepSet
        \item Testing on LHCb data
      \end{itemize}
    \item Conclusion and Outlook
  \end{itemize}

\end{frame}

\begin{frame}{Einführung}
  \tableofcontents
\end{frame}

\section{Fonts}
\begin{frame}
  Der Font der im Corporate Design der TU Dortmund vorgesehen ist,
  ist \enquote{Akkurat Office}.

  Falls dieser nicht verfügbar ist, wird als Alternative \enquote{Fira Sans}
  verwendet.

  Für Mathematik wird bei Verwendung von \texttt{xelatex} oder \texttt{lualatex} der Font \enquote{Fira Math} verwendet.
\end{frame}

\begin{frame}{Mathe}
  \begin{align*}
    \nabla \cdot \symbf{B} &= 0 &
    \nabla \cdot \symbf{E} &= \frac{ρ}{ε_0} \\
    \nabla \times \symbf{E} &= -\partial_t \symbf{B} &
    \nabla \times \symbf{B} &= μ_0 \symbf{j} + μ_0 ε_0 \partial_t \symbf{E} &
  \end{align*}
\end{frame}
\end{document}