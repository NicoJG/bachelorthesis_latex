% This example is meant to be compiled with lualatex or xelatex
% The theme itself also supports pdflatex
\PassOptionsToPackage{unicode}{hyperref}
\documentclass[aspectratio=1610, 10pt]{beamer}

% Load packages you need here
\usepackage[ngerman, english]{babel}

\usepackage[autostyle]{csquotes}

\usepackage{amsmath}
\usepackage{amssymb}
\usepackage{mathtools}
%\input{lhcb-symbols-def.tex}

% Enable Unicode-Math and follow the ISO-Standards for typesetting math
\usepackage{unicode-math}

\usepackage[
  separate-uncertainty=true,
  per-mode=symbol-or-fraction,
]{siunitx}
% automatically choose the right locale
\addto\extrasngerman{\sisetup{locale = DE}}
\addto\extrasenglish{\sisetup{locale = UK}}

\usepackage{graphicx}
%\usepackage{enumitem}
\usepackage{booktabs}

\usepackage{hyperref}
\usepackage{bookmark}

\usepackage{animate}

% load the theme after all packages

\usetheme[
  showtotalframes, % show total number of frames in the footline
]{tudo}

% Put settings here, like
\unimathsetup{
  math-style=ISO,
  bold-style=ISO,
  sans-style=italic,
  nabla=upright,
  partial=upright,
  warnings-off={mathtools-colon,mathtools-overbracket}, % suppress some unnecessary warnings
}

\setbeamertemplate{blocks}[rounded][shadow=false]

\title{Development of a multivariate algorithm for the classification of B~mesons at the LHCb experiment}
\author[N.~Guth]{Nico Guth}
\institute[AG Albrecht]{Arbeitsgruppe Albrecht \\ Fakultät Physik}
\date{Bachelor talk, 20.07.2022}
%\subtitle{Bachelorthesis}
%\titlegraphic{\includegraphics[width=0.7\textwidth]{images/tudo-title-2.jpg}}

\begin{document}

\maketitle

\section*{Introduction}

\begin{frame}
  \begin{block}{Goal of my thesis:}
    Develop an algorithm that distinguishes between $B^0_d$ and $B^0_s$~mesons based on tracks associated with the signal $B$~meson without tracks of the signal decay. (in $pp$-collisions at the LHCb detector) 
  \end{block}

  \pause
  \medskip
  \centering
  \begin{columns}
    \begin{column}{0.6\textwidth}
      \textbf{Structure of this talk:}
        \begin{itemize}
          \item Motivation
          \item $B$ meson production in $pp$-collisions
          \item The LHCb detector
          \item Development of a $B$ meson classifier
          \begin{itemize}
            \item Identification of same side tracks using a BDT
            \item Classification of the $B$ meson using a DeepSet
            \item Testing on real LHCb data
          \end{itemize}
          \item Conclusion and Outlook
        \end{itemize}
    \end{column}
  \end{columns}
\end{frame}

\begin{frame}{Motivation}
  \begin{itemize}
    \item \underline{support background reduction} where $B^0_d \: \left(\bar{b}d\right)$ or $B^0_s \: \left(\bar{b}s\right)$ is unwanted
    \begin{itemize}
      \item \textit{partial backgrounds} with missing information in the signal decay
      \item backgrounds with similar signal kinematics
      \item e.g. $B^0_s \rightarrow D^+_s K^-$ with $B^0_d$ backgrounds in the signal region
    \end{itemize}
    \pause
    \medskip
    \item excluding the signal decay \\$\rightarrow$ independence of the signal decay channel
    \item associated event contains enough information (in principle)
    \begin{itemize}
      \item mass difference of $B^0_d$ and $B^0_s$ ($\qty{87}{\MeV}$)
      \item different fragmentation processes
    \end{itemize} 
  \end{itemize}
\end{frame}

\section*{Theoretical and experimental background}

\begin{frame}{$B$ meson production in $pp$-collisions}
  \begin{columns}
    \begin{column}{0.6\textwidth}
      \centering
      \includegraphics[width=\textwidth]{images/FlavourTaggingScheme.pdf}

      \tiny \url{https://twiki.cern.ch/twiki/bin/view/LHCb/FlavourTaggingConferencePlots}
    \end{column}
    \begin{column}{0.4\textwidth}
      \begin{itemize}
        \item $pp$-collisions produce many particles
        \item gluon-fusion may lead to a $b\bar{b}$-pair
        \item hadronisation $\rightarrow$ $B$ meson and fragmentation particles
        \item Lorentz boosted signal $B$ $\rightarrow$ distinguish secondary from primary vertex
        \item for $B^0_d$ vs $B^0_s$ only same side (SS) relevant
        \item here: exclude the signal decay
      \end{itemize}
    \end{column}
  \end{columns}
\end{frame}

\begin{frame}{The LHCb detector}
  \centering
  \includegraphics[height=0.9\textheight]{images/lhcb_detector.png}

  \tiny \url{https://iopscience.iop.org/article/10.1088/1748-0221/3/08/S08005}
\end{frame}

\section*{Development of a B meson classifier}

\begin{frame}{Development of a $B$ meson classifier}
  \centering
  \begin{columns}
    \begin{column}{0.6\textwidth}
      \textbf{Strategy:}
      \begin{itemize}
        \item same side track identification using a BDT
        \item $B$ meson classification using a DeepSet
        \item test on real LHCb data
      \end{itemize}

      \pause
      \textbf{Training dataset:}
      \begin{itemize}
        \item training with LHCb simulation
        \item combined dataset:
        \begin{itemize}
          \item $B^0_d \rightarrow J/\psi K^*$
          \item $B^0_s \rightarrow D_s^+ \pi^-$
        \end{itemize}
        \item found differences by year and simulation version \\$\rightarrow$ chose 2016 and same simulation version
        \item dataset contains 0.4 million events and 18 million tracks
      \end{itemize}
    \end{column}
  \end{columns}
\end{frame}

\section*{SS track identification using a BDT}

\begin{frame}{Boosted Decision Tree (BDT)}
  \begin{columns}
    \begin{column}{0.5\textwidth}
      \centering
      \textbf{Simple Decision Tree:}
      \includegraphics[width=\textwidth]{images/decision_tree.png}

      \tiny \url{https://arxiv.org/abs/physics/0703039}
    \end{column}
    \pause
    \begin{column}{0.5\textwidth}
      \textbf{Boosted Decision Tree:}
      \begin{itemize}
        \item ensemble of multiple small Decision Trees
        \item weighted sum transformed with logistic function \\$\rightarrow$ estimated class probabilities
        \item iterative training through gradient boosting \\$\rightarrow$ minimum of a loss function
      \end{itemize}
    \end{column}
  \end{columns}
\end{frame}

\begin{frame}{SS track identification: Feature Selection}
  \begin{columns}
    \begin{column}{0.3\textwidth}
      \centering
      \begin{tabular}{c c}
        \toprule
        \multicolumn{2}{c}{track features} \\
        \midrule
        $p_\text{T}$        & $\text{IP}_\text{SV}$ \\ 
        $p_\text{proj}$     & $\chi^2(\text{IP}_\text{SV})$ \\ 
        $\Delta p_\text{T}$ & $\sigma(\text{IP}_\text{pileup vtx})$ \\ 
        $\Delta z$          & $\text{IP}_\text{best PV}$ \\    
        $\Delta \eta$       & $\chi^2(\text{IP}_\text{best PV})$ \\ 
        $\cos(\Delta \phi)$ & $\text{IP}_\text{min}$ \\ 
        $\text{Prob}_\text{ghost}$ & same PV \\
        $\chi^2(\text{vtx})$     & cone isolation \\
        SumBDT              & $N_\text{non iso}$ \\ 
        MinBDT              & $\sum p_\text{in cone}$ \\ 
        SumMinBDT           &  \\
        \bottomrule
      \end{tabular}
    \end{column}
    \begin{column}{0.7\textwidth}
      \centering
      \includegraphics[width=\textwidth]{images/SS_feature_importances.pdf}
    \end{column}
  \end{columns}
\end{frame}

\begin{frame}{SS track identification: BDT training and results}
  \begin{columns}
    \begin{column}{0.5\textwidth}
      \centering
      \only<1>{%
        \textbf{Error rate during training}
        \includegraphics[width=0.85\textwidth]{images/SS_history_error.pdf}
      }
      \only<2-3>{%
        \textbf{Distribution of $\text{Prob}_\text{SS}$}
        \includegraphics[width=0.85\textwidth]{images/SS_output.pdf}
      }
    \end{column}
    \begin{column}{0.5\textwidth}
      \only<1-2>{%
        \begin{itemize}
          \item 60\% training data, 40\% test data
          \item 2000 decision trees with maximum tree depth of 4
          \item loss: logistic regression for binary classification
          \item output: $\text{Prob}_\text{SS} \in [0,1]$
        \end{itemize}
      }
      \only<3>{%
        \centering
        \textbf{ROC curve of the BDT predictions}
        \includegraphics[width=0.85\textwidth]{images/SS_ROC.pdf}
      }
    \end{column}
  \end{columns}
\end{frame}

\section*{B meson classification using a DeepSet}

\begin{frame}{Neural Network (NN)}
  \begin{columns}
    \begin{column}{0.5\textwidth}
      \centering
      \includegraphics[width=0.85\textwidth]{images/NN_schematic.png}

      \tiny \url{https://www.knime.com/blog/a-friendly-introduction-to-deep-neural-networks}
    \end{column}
    \begin{column}{0.5\textwidth}
      \begin{itemize}
        \item non-linear transformation $\vec{x} \rightarrow \vec{y}$
        \item multiple steps called layers of activation \\$\rightarrow$ $\vec{a}^{(n)} = f^{(n)}\left( W^{(n)} \cdot \vec{a}^{(n-1)} + \vec{b}^{n} \right)$
        \item activation functions used here:
        \begin{itemize}
          \item $f_\text{ReLU}(z) = \max (0, z)$
          \item $f_\text{Sigmoid}(z) = \frac{1}{1+e^{-z}}$
        \end{itemize}
        \item iterative training through backpropagation (gradient descent)
      \end{itemize}
    \end{column}
  \end{columns}
\end{frame}

\begin{frame}{DeepSet}
  \begin{columns}
    \begin{column}{0.5\textwidth}
      \begin{itemize}
        \item extension of NNs to allow inputs of sets of vectors \\
        $\rightarrow$ variable input length \\
        $\rightarrow$ permutation invariant
        \item $\displaystyle f(X) = \rho \left( \sum_{x_i \in \text{ \normalsize{$X$}}} \phi(x_i) \right)$
      \end{itemize}

      \medskip
      \centering
      \includegraphics[width=0.9\textwidth]{images/DeepSet_schematic.png}

      \tiny \url{https://arxiv.org/abs/1703.06114}
    \end{column}
    \begin{column}{0.5\textwidth}
      \pause
      \textbf{DeepSet for B meson classification:}
      \begin{itemize}
        \item one set $X$ per event
        \item one vector $x_i$ per track
        \item $\phi$-network layer sizes: 23, 64, 128, 64
        \item $\rho$-network layer sizes: 64, 128, 64, 1
        \item $f_\text{ReLU}$ for hidden layers  
        \item $f_\text{Sigmoid}$ for the output layer
        \item output: $\text{Prob}_{B_s} \in [0,1]$
      \end{itemize}
    \end{column}
  \end{columns}
\end{frame}

\begin{frame}{B meson classification: Feature Selection}
  \begin{columns}
    \begin{column}{0.25\textwidth}
      \centering
      \begin{tabular}{c c}
        \toprule
        \multicolumn{2}{c}{track features} \\
        \midrule
        $p$                 & $\text{\textbf{Prob}}_\text{\textbf{SS}}$ \\ %"Tr_T_P","Tr_ProbSS"
        $p_\text{T}$        & $\text{Prob}_e$ \\ %"Tr_T_PT", "Tr_T_PROBNNe"
        $p_\text{proj}$     & $\text{Prob}_\text{ghost}$ \\ %"Tr_p_proj","Tr_T_PROBNNghost"
        $\Delta p$          & $\text{Prob}_K$ \\ %"Tr_diff_p", "Tr_T_PROBNNk"
        $\Delta p_\text{T}$ & $\text{Prob}_\mu$ \\ %"Tr_diff_pt","Tr_T_PROBNNmu"
        $\Delta z$          & $\text{Prob}_p$ \\ %"Tr_diff_z", "Tr_T_PROBNNp"
        $\cos(\Delta \phi)$ & $\text{Prob}_\pi$ \\ %"Tr_cos_diff_phi", "Tr_T_PROBNNpi"
        $\Delta \eta$       & $\sigma(\text{IP}_\text{pileup vtx})$ \\ %"Tr_diff_eta", "Tr_T_IP_trPUS"
        $\text{IP}_\text{SV}$        & $Q_\text{VELO}$ \\ %"Tr_T_IP_trMother",  "Tr_T_VeloCharge"
        $\chi^2(\text{IP}_\text{SV})$    & SumBDT \\ %"Tr_T_IPCHI2_trMother", "Tr_T_SumBDT_ult"
        $\text{IP}_\text{min}$               & MinBDT \\ %"Tr_T_MinIP", "Tr_T_MinBDT_ult"
        $\chi^2(\text{IP}_\text{min})$           & \\ %"Tr_T_MinIPChi2"
        \bottomrule
    \end{tabular}
    \end{column}
    \begin{column}{0.75\textwidth}
      \centering
      \includegraphics[width=\textwidth]{images/B_feature_importances.pdf}
    \end{column}
  \end{columns}
\end{frame}

\begin{frame}{B meson classification: DeepSet training and results}
  
  \begin{columns}
    \begin{column}{0.5\textwidth}
      \centering
      \only<1>{%
        \textbf{Error rate during training}
        \includegraphics[width=0.85\textwidth]{images/B_history_error.pdf}
      }
      \only<2-3>{%
        \textbf{Distribution of $\text{Prob}_{B_s}$}
        \includegraphics[width=0.85\textwidth]{images/B_output.pdf}
      }
    \end{column}
    \begin{column}{0.5\textwidth}
      \only<1-2>{%
        \begin{itemize}
          \item 60\% training data, 40\% test data (standard scaled)
          \item regularisation: 
          \begin{itemize}
            \item early stopping after 50 iterations
            \item Dropout of 50\%
          \end{itemize}
          \item loss: binary cross entropy
          \item optimizer: Adam
          \item output: $\text{Prob}_{B_s} \in [0,1]$
        \end{itemize}
      }
      \only<3>{%
        \centering
        \textbf{ROC curve of the DeepSet predictions}
        \includegraphics[width=0.85\textwidth]{images/B_ROC.pdf}
      }
    \end{column}
  \end{columns}
\end{frame}

\section*{Testing on LHCb data}

\begin{frame}{Testing on LHCb data: Overview}
  \begin{columns}
    \begin{column}{0.5\textwidth}
      \begin{itemize}
        \item run 2 LHCb data selected for $B^0_d \, \text{or} \, B^0_s \rightarrow J/\psi K^0_S$
        \item testing strategy:
        \begin{itemize}
          \item apply the developed algorithm on every event
          \item selections with $\text{Prob}_{B_s} \leq x$ and $\text{Prob}_{B_s} \geq x$
          \item fit the resulting mass distributions
          \item estimate $B^0_d$ and $B^0_s$ counts by integration
        \end{itemize}
      \end{itemize}
    \end{column}
    \pause
    \begin{column}{0.5\textwidth}
      \centering
      \textbf{Example fit of the data:\\(peaks at $M_{B_d} = \qty{5280}{\MeV}$ and $M_{B_s} = \qty{5367}{\MeV}$)}
      \includegraphics[width=0.9\textwidth]{images/fit_example.pdf}
    \end{column}
  \end{columns}  
\end{frame}

\begin{frame}{Testing on LHCb data: Background reduction}
  \begin{columns}
    \begin{column}{0.5\textwidth}
      \begin{itemize}
        \item trained BDT with 13 features on $B^0_d \rightarrow J/\psi K^0_S$ simulation + combinatorial background from the data
        \item achieved ROC AUC of 0.989
        \item manually: misidentification of $K^0_S$ as $\Lambda^0$ or $K^*$
      \end{itemize}
    \end{column}
    \begin{column}{0.5\textwidth}
      \centering
      \textbf{Signal B mass after background reduction:}
      \includegraphics[width=0.9\textwidth]{images/BKG_reduced.pdf}
    \end{column}
  \end{columns}
\end{frame}

\begin{frame}{Testing on LHCb data: Results (ratio)}
  \begin{columns}
    \begin{column}{0.5\textwidth}
      \centering
      \includegraphics[width=0.9\textwidth]{images/data_ratio.pdf}
    \end{column}
    \begin{column}{0.5\textwidth}
      \begin{itemize}
        \item without separation: constant ratio $n_{B_s}/n_{B_d}$
        \item expected value: \\
        \begin{equation*}
          \frac{\text{BR}(B_s \rightarrow J/\psi \, K^0_\text{S})}{\text{BR}(B_d \rightarrow J/\psi \, K^0_\text{S})} \cdot f_s/f_d(\qty{13}{\TeV}) = \num{0.0109\pm0.0010}
        \end{equation*}
        \item $\text{Prob}_{B_s} \leq x$ starts with low statistics in the $B_s$ mode and is then almost constant
        \item $\text{Prob}_{B_s} \geq x$ starts constant and then shows an increase 
        \item indication of some separation
        \item all ratios compatible with the expected value
      \end{itemize}
    \end{column}
  \end{columns}
\end{frame}

\begin{frame}{Testing on LHCb data: Results (efficiencies, similar to a ROC curve)}
  \begin{columns}
    \begin{column}{0.5\textwidth}
      \centering
      \includegraphics[width=0.9\textwidth]{images/data_roc.pdf}
    \end{column}
    \begin{column}{0.5\textwidth}
      \begin{itemize}
        \item calculated efficiencies $\varepsilon_B = n_B(x)/n_B(\text{no cut})$
        \item plot $\varepsilon_{B_d}$ against $\varepsilon_{B_s}$
        \item should be similar to a ROC curve
        \item no clear separation visible
      \end{itemize}
    \end{column}
  \end{columns}
\end{frame}

\begin{frame}{Animation of $n_{B_s}/n_{B_d}$ and the corresponding fits}
  %\animategraphics[loop,controls,palindrome,width=\linewidth]{5}{images/ratio_animation/frame_}{0}{48}
\end{frame}

\section*{Conclusion and outlook}

\begin{frame}{Conclusion and outlook}
  \textbf{Results:}
  \begin{itemize}
    \item BDT can identify SS tracks (ROC AUC: 0.76) and helps the DeepSet (feature importances)
    \item DeepSet achieves clear separation of $B^0_d$ and $B^0_s$ events on the simulated test sample (ROC AUC: 0.74)
    \item separation on LHCb data cannot be quantified
    \begin{itemize}
      \item some level of separation seen in $n_{B_s}/n_{B_d}$
      \item no sign of separation in ROC curve
    \end{itemize}
    \item reasons for performance loss unknown (selection differences? mismodeled simulation?)
  \end{itemize}
  
  \pause
  \medskip
  \textbf{Outlook and suggestions:}
  \begin{itemize}
    \item indications that a similar approach could succeed
    \item compare simulation kinematics with real world kinematics
    \item ensure that kinematic differences originate from the mass difference
    \item potentially equalize the kinematics by reweighting the training data
    \item if successful, extension to include charged $B$ mesons possible
  \end{itemize}
\end{frame}

\appendix

\begin{frame}[plain]
  \centering
  \begin{beamercolorbox}[center, wd=\textwidth]{title}
    \textcolor{tugreen}{\rule{\textwidth}{1pt}}\\[0.5\baselineskip]%
    \usebeamerfont{title}Thank you for your attention!
    \\[0.5\baselineskip]%
    \usebeamerfont{subtitle}Here is some art I found in the data:\newline%
    \textcolor{tugreen}{\rule{\textwidth}{1pt}}%
  \end{beamercolorbox}%
  \begin{columns}
    \begin{column}{0.5\textwidth}
      \centering
      \includegraphics[width=0.7\textwidth]{images/backup/art1.png}
    \end{column}
    \begin{column}{0.5\textwidth}
      \centering
      \includegraphics[width=0.7\textwidth]{images/backup/art2.png}
    \end{column}
  \end{columns}
\end{frame}

\begin{frame}[plain]
  \centering
  \begin{beamercolorbox}[center, wd=\textwidth]{title}
    \textcolor{tugreen}{\rule{\textwidth}{1pt}}\\[0.5\baselineskip]%
    \usebeamerfont{title}Backup
    \textcolor{tugreen}{\rule{\textwidth}{1pt}}%
  \end{beamercolorbox}%
\end{frame}

\begin{frame}{Correlation Matrix}
  \centering
  \includegraphics[height=0.9\textheight]{images/backup/correlation_matrix.pdf}
\end{frame}

\begin{frame}{Background BDT}
  \begin{columns}
    \begin{column}{0.33\textwidth}
      \centering
      \begin{tabular}{c c}
        \toprule
        \multicolumn{2}{c}{signal features} \\
        \midrule
        IP$(B^0)$                   & $p_\text{T}(\pi^+)$ \\% "B_IP_OWNPV","piplus_PT"
        IP$(J/\psi)$                & $p_\text{T}(\pi^-)$ \\% "Jpsi_IP_OWNPV","piminus_PT"
        IP$(K^0_\text{S})$          & $p_\text{T}(K^0_\text{S})$ \\% "KS0_IP_OWNPV","KS0_PT"
        IP$(\mu^+)$                 & $\eta(B^0)$ \\% "muplus_IP_OWNPV","B_LOKI_ETA"
        IP$(\mu^-)$                 & $\eta(K^0_\text{S})$ \\% "muminus_IP_OWNPV","KS0_LOKI_ETA"
        FD$(K^0_\text{S})$    & $p_z(K^0_\text{S})$ \\% "KS0_FD_OWNPV","KS0_PZ"
        $\chi^2(\text{fit})$  & \\% "B_LOKI_DTF_CHI2NDOF",
        \bottomrule
    \end{tabular}
    \end{column}
    \begin{column}{0.33\textwidth}
      \centering
      \includegraphics[width=\textwidth]{images/backup/bkg_output.pdf}
    \end{column}
    \begin{column}{0.33\textwidth}
      \centering
      \includegraphics[width=\textwidth]{images/backup/bkg_roc.pdf}
    \end{column}
  \end{columns}
\end{frame}

\begin{frame}{Test on LHCb data: DeepSet output}
  \begin{columns}
    \begin{column}{0.5\textwidth}
      \centering
      \includegraphics[width=\textwidth]{images/backup/data_B_classifier_output.pdf}
    \end{column}
    \begin{column}{0.5\textwidth}
      \centering
      \includegraphics[width=\textwidth]{images/backup/data_cut_comparison.pdf}
    \end{column}
  \end{columns}
\end{frame}

\begin{frame}{Fit of the $B_d$ mode on the simulated sample:}
  \centering
  \includegraphics[width=0.5\textwidth]{images/fit_mc.pdf}
\end{frame}

\begin{frame}{Fit functions}
  \begin{equation*}
    F(M_B) = N_\text{bkg} \cdot F_\text{bkg}(M_B) + N_{B_d} \cdot F_{B_d}(M_B) + N_{B_s} \cdot F_{B_s}(M_B)
  \end{equation*}
  \begin{align*}
    F_\text{bkg}(M_B) = \exp(-\lambda \cdot M_B) \, .
  \end{align*}
  \begin{align*}
    F_B(M_B) = &f_1 \cdot f_2 \cdot F_\text{CB}\left(\frac{M_B-\mu}{\sigma_1}, \beta_1, m_1\right) \nonumber\\
                &+ (1-f_1) \cdot f_2 \cdot F_\text{CB}\left(-\frac{M_B-\mu}{\sigma_2}, \beta_2, m_2\right) \nonumber\\
                &+ (1-f_1) \cdot (1-f_2) \cdot F_\text{gauss}\left(M_B,\mu,\sigma_3\right) \, , \label{eqn:FB}
  \end{align*}
  \begin{equation*}
    F_\text{CB}(x,\beta,m) = 
    \begin{cases}
        N \cdot \exp(-\frac{x^2}{2}) & \text{for } x > -\beta \\
        N \cdot \left(\frac{m}{|\beta|}\right)^m \cdot \exp\left(-\frac{\beta^2}{2}\right) \cdot \left(\frac{m}{|b|}-|b| - x\right)^{-m} & \text{for } x \leq -\beta
    \end{cases}
  \end{equation*}
  \begin{equation*}
    F_\text{gauss}\left(x,\mu,\sigma\right) = \frac{1}{\sqrt{2}\pi\sigma} \cdot \exp\left(-\frac{1}{2}\left(\frac{x-\mu}{\sigma}\right)^2\right)
  \end{equation*}
\end{frame}

\begin{frame}{The Standard Model of particle physics}
  \centering
  \includegraphics[height=0.9\textheight]{images/standard_model.pdf}

  \tiny \url{https://en.wikipedia.org/wiki/Standard_Model}
\end{frame}

\end{document}