% This example is meant to be compiled with lualatex or xelatex
% The theme itself also supports pdflatex
\PassOptionsToPackage{unicode}{hyperref}
\documentclass[aspectratio=1610, 10pt]{beamer}

% Load packages you need here
\usepackage[ngerman, english]{babel}

\usepackage[autostyle]{csquotes}

\usepackage{amsmath}
\usepackage{amssymb}
\usepackage{mathtools}
%\input{lhcb-symbols-def.tex}

% Enable Unicode-Math and follow the ISO-Standards for typesetting math
\usepackage{unicode-math}

\usepackage[
  separate-uncertainty=true,
  per-mode=symbol-or-fraction,
]{siunitx}
% automatically choose the right locale
\addto\extrasngerman{\sisetup{locale = DE}}
\addto\extrasenglish{\sisetup{locale = UK}}

\usepackage{graphicx}
%\usepackage{enumitem}
\usepackage{booktabs}

\usepackage{hyperref}
\usepackage{bookmark}

% load the theme after all packages

\usetheme[
  showtotalframes, % show total number of frames in the footline
]{tudo}

% Put settings here, like
\unimathsetup{
  math-style=ISO,
  bold-style=ISO,
  sans-style=italic,
  nabla=upright,
  partial=upright,
  warnings-off={mathtools-colon,mathtools-overbracket}, % suppress some unnecessary warnings
}

\title{Development of a multivariate algorithm for the classification of B~mesons at the LHCb experiment}
\author[N.~Guth]{Nico Guth}
\institute[AG Albrecht]{Arbeitsgruppe Albrecht \\ Fakultät Physik}
\date{Bachelor talk, 20.07.2022}
%\subtitle{Bachelorthesis}
%\titlegraphic{\includegraphics[width=0.7\textwidth]{images/tudo-title-2.jpg}}

\begin{document}

\maketitle

\section*{Introduction}

\begin{frame}{Overview}
  \begin{block}{Goal of my thesis:}
    Develop an algorithm that distinguishes between $B^0_d$ and $B^0_s$~mesons based on tracks associated with the signal $B$~meson without tracks of the signal decay. (in $pp$-collisions at the LHCb detector) 
  \end{block}

  \medskip
  \textbf{Structure of this talk:}
  \begin{itemize}
    \item Motivation
    \item Theoretical and experimental background
      \begin{itemize}
        \item The Standard Model
        \item $B$ meson production in $pp$-collisions
        \item The LHCb detector
      \end{itemize}
    \item Development of the $B$ meson classifier
      \begin{itemize}
        \item Identification of same side tracks using a BDT
        \item Classification of the $B$ meson using a DeepSet
        \item Testing on LHCb data
      \end{itemize}
    \item Conclusion and Outlook
  \end{itemize}

\end{frame}

\begin{frame}{Motivation}
  \begin{itemize}
    \item at LHCb: Analysis of $pp$-collisions involving $b$ or $c$ quarks, e.g. $B^0_d \: (bd)$ or $B^0_s \: (bs)$
    \item independence of signal decay channel
    \item support background reduction of the other $B$ meson type
    \begin{itemize}
      \item \textit{partial backgrounds} with missing information in the signal decay
      \item backgrounds with similar signal kinematics
      \item e.g. $B^0_s \rightarrow D^+_s K^-$ with $B^0_d$ backgrounds in the signal region
    \end{itemize}
  \end{itemize}
\end{frame}

\section*{Theoretical and experimental background}

\begin{frame}{The Standard Model of particle physics}
  \centering
  \includegraphics[height=0.9\textheight]{images/standard_model.pdf}

  \tiny \url{https://en.wikipedia.org/wiki/Standard_Model}
\end{frame}

\begin{frame}{$B$ meson production in $pp$-collisions}
  \begin{columns}
    \begin{column}{0.6\textwidth}
      \centering
      \includegraphics[width=\textwidth]{images/FlavourTaggingScheme.pdf}

      \tiny \url{https://twiki.cern.ch/twiki/bin/view/LHCb/FlavourTaggingConferencePlots}
    \end{column}
    \begin{column}{0.4\textwidth}
      \begin{itemize}
        \item $pp$-collisions produce many particles
        \item gluon-fusion may lead to a $b\bar{b}$-pair
        \item hadronisation $\rightarrow$ $B$ meson and fragmentation particles
        \item Lorentz boosted signal $B$ $\rightarrow$ distinguish secondary from primary vertex
        \item for $B^0_d$ vs $B^0_s$ only same side (SS) relevant
        \item here: exclude the signal decay
      \end{itemize}
    \end{column}
  \end{columns}
\end{frame}

\begin{frame}{The LHCb detector}
  \begin{columns}
    \begin{column}{0.6\textwidth}
      \centering
      \includegraphics[width=\textwidth]{images/lhcb_detector.png}

      \tiny \url{https://iopscience.iop.org/article/10.1088/1748-0221/3/08/S08005}
    \end{column}
    \begin{column}{0.4\textwidth}
      \begin{itemize}
        \item single arm forward-spectrometer at the LHC %($\qty{10}{\milli \rad}$ to $\qty{300}{\milli \rad}$)
        \item particle tracking: VELO, tracking stations
        \item particle identification: magnet, RICH, calorimeters, muon chambers
      \end{itemize}
    \end{column}
  \end{columns}
\end{frame}

\section*{Development of the B meson classifier}

\begin{frame}{Development of the $B$ meson classifier}
  \centering
  \begin{columns}
    \begin{column}{0.6\textwidth}
      \begin{itemize}
        \item training with LHCb simulation data
        \item combining $B^0_d \rightarrow J/\psi K^*$ and $B^0_s \rightarrow D_s^+ \pi^-$
        \item found differences by year and simulation version \\$\rightarrow$ choose 2016 and same simulation version
        \item dataset contains 18 million tracks and 0.4 million events
        \item highly correlated features are reduced 
      \end{itemize}

      \textbf{Strategy:}
      \begin{itemize}
        \item same side track identification using a BDT
        \item $B$ meson classification using a DeepSet
        \item test on real LHCb data
      \end{itemize}
    \end{column}
  \end{columns}
\end{frame}

\section*{SS track identification using a BDT}

\begin{frame}{Boosted Decision Tree (BDT)}
  \begin{columns}
    \begin{column}{0.5\textwidth}
      \centering
      \textbf{Simple Decision Tree:}
      \includegraphics[width=\textwidth]{images/decision_tree.png}

      \tiny \url{https://arxiv.org/abs/physics/0703039}
    \end{column}
    \begin{column}{0.5\textwidth}
      \textbf{Boosted Decision Tree:}
      \begin{itemize}
        \item ensemble of multiple small Decision Trees
        \item weighted sum transformed with logistic function \\$\rightarrow$ estimated class probabilities
        \item iterative training through gradient boosting \\$\rightarrow$ minimum of the loss function
      \end{itemize}
    \end{column}
  \end{columns}
\end{frame}

\begin{frame}{SS track identification: Feature Selection}
  \begin{columns}
    \begin{column}{0.35\textwidth}
      \centering
      \begin{tabular}{c c}
        \toprule
        track- & track- \\
        feature & feature \\
        \midrule
        $p_\text{T}$        & $\text{IP}_\text{SV}$ \\ 
        $p_\text{proj}$     & $\chi^2(\text{IP}_\text{SV})$ \\ 
        $\Delta p_\text{T}$ & $\sigma(\text{IP}_\text{pileup vtx})$ \\ 
        $\Delta z$          & $\text{IP}_\text{best PV}$ \\    
        $\Delta \eta$       & $\chi^2(\text{IP}_\text{best PV})$ \\ 
        $\cos(\Delta \phi)$ & $\text{IP}_\text{min}$ \\ 
        $\text{Prob}_\text{ghost}$ & same PV \\
        $\chi^2(\text{vtx})$     & cone isolation \\
        SumBDT              & $N_\text{non iso}$ \\ 
        MinBDT              & $\sum p_\text{in cone}$ \\ 
        SumMinBDT           &  \\
        \bottomrule
      \end{tabular}
    \end{column}
    \begin{column}{0.65\textwidth}
      \centering
      \includegraphics[width=\textwidth]{images/SS_feature_importances.pdf}
    \end{column}
  \end{columns}
\end{frame}

\begin{frame}{SS track identification: BDT training}
  \begin{columns}
    \begin{column}{0.4\textwidth}
      \centering
      \includegraphics[width=0.95\textwidth]{images/SS_history_error.pdf}
    \end{column}
    \begin{column}{0.6\textwidth}
      \begin{itemize}
        \item 60\% training data, 40\% test data
        \item 2000 decision trees with maximum tree depth of 4
        \item loss: logistic regression for binary classification
        \item output: $\text{Prob}_\text{SS} \in [0,1]$
      \end{itemize}
    \end{column}
  \end{columns}
\end{frame}

\begin{frame}{SS track identification: Results}
  \begin{columns}
    \begin{column}{0.5\textwidth}
      \centering
      \textbf{Distribution of $\text{Prob}_\text{SS}$}
      \includegraphics[width=0.85\textwidth]{images/SS_output.pdf}
    \end{column}
    \begin{column}{0.5\textwidth}
      \centering
      \textbf{ROC curve of the BDT predictions}
      \includegraphics[width=0.85\textwidth]{images/SS_ROC.pdf}
    \end{column}
  \end{columns}
\end{frame}
\end{document}