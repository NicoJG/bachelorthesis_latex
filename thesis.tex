%%%%%%%%%%%%%%%%%%%%%%%%%%%%%%%%%%%%%%%%%%%%%%%%%%%%%%%%%%%%%%%%%%%%%%%%%%%%%%%%
%%%%%%%%%%%%%%%%%%   Vorlage für eine Abschlussarbeit   %%%%%%%%%%%%%%%%%%%%%%%%
%%%%%%%%%%%%%%%%%%%%%%%%%%%%%%%%%%%%%%%%%%%%%%%%%%%%%%%%%%%%%%%%%%%%%%%%%%%%%%%%

% Erstellt von Maximilian Nöthe, <maximilian.noethe@tu-dortmund.de>
% ausgelegt für lualatex und Biblatex mit biber

% Kompilieren mit
% latexmk --lualatex --output-directory=build thesis.tex
% oder einfach mit:
% make

\documentclass[
  tucolor,       % remove for less green,
  BCOR=12mm,     % 12mm binding corrections, adjust to fit your binding
  parskip=half,  % new paragraphs start with half line vertical space
  open=any,      % chapters start on both odd and even pages
  cleardoublepage=plain,  % no header/footer on blank pages
]{tudothesis}

% Package for insertion of line numbers
\usepackage{lineno}

% Warning, if another latex run is needed
\usepackage[aux]{rerunfilecheck}

% just list chapters and sections in the toc, not subsections or smaller
\setcounter{tocdepth}{1}

%------------------------------------------------------------------------------
%------------------------------ Fonts, Unicode, Language ----------------------
%------------------------------------------------------------------------------
\usepackage{fontspec}
\defaultfontfeatures{Ligatures=TeX}  % -- becomes en-dash etc.

% load all used languages
% and set the main language of this thesis
% use this if this thesis is written in German:
%\usepackage[main=ngerman, english]{babel}
% use this if this thesis is written in English:
\usepackage[main=english, ngerman]{babel}

% intelligent quotation marks, language and nesting sensitive
\usepackage[autostyle]{csquotes}

% microtypographical features, makes the text look nicer on the small scale
\usepackage{microtype}

%------------------------------------------------------------------------------
%------------------------ Math Packages and settings --------------------------
%------------------------------------------------------------------------------

\usepackage{amsmath}
\usepackage{amssymb}
\usepackage{mathtools}

% Enable Unicode-Math and follow the ISO-Standards for typesetting math
\usepackage[
  math-style=ISO,
  bold-style=ISO,
  sans-style=italic,
  nabla=upright,
  partial=upright,
  warnings-off={mathtools-colon,mathtools-overbracket}, % suppress some unnecessary warnings
]{unicode-math}
\setmathfont{Latin Modern Math}

% nice, small fracs for the text with \sfrac{}{}
\usepackage{xfrac}


%------------------------------------------------------------------------------
%---------------------------- Numbers and Units -------------------------------
%------------------------------------------------------------------------------

\usepackage[
  separate-uncertainty=true,
  per-mode=symbol-or-fraction,
]{siunitx}
% automatically choose the right locale
\addto\extrasngerman{\sisetup{locale = DE}}
\addto\extrasenglish{\sisetup{locale = UK}}

%------------------------------------------------------------------------------
%-------------------------------- tables  -------------------------------------
%------------------------------------------------------------------------------

\usepackage{booktabs}       % \toprule, \midrule, \bottomrule, etc
\usepackage{multirow}       % multiple row spanning cells

%------------------------------------------------------------------------------
%-------------------------------- graphics -------------------------------------
%------------------------------------------------------------------------------

\usepackage{graphicx}
% currently broken
% \usepackage{grffile}

% allow figures to be placed in the running text by default:
\usepackage{scrhack}
\usepackage{float}
\floatplacement{figure}{htbp}
\floatplacement{table}{htbp}

% keep figures and tables in the section
\usepackage[section, below]{placeins}

% allows to include PDFs as full pages
\usepackage{pdfpages}

% Set the PDF Version of this document to 1.7 (1.4 is the current default)
% This is needed so that PDFs with Version >1.5 can be included
\pdfvariable minorversion=7

%------------------------------------------------------------------------------
%---------------------- customize list environments ---------------------------
%------------------------------------------------------------------------------

\usepackage{enumitem}

%------------------------------------------------------------------------------
%------------------------------ Bibliographie ---------------------------------
%------------------------------------------------------------------------------

\usepackage[
  backend=biber,   % use modern biber backend
  autolang=hyphen, % load hyphenation rules for if language of bibentry is not
                   % german, has to be loaded with \setotherlanguages
                   % in the references.bib use langid={en} for english sources
  sorting=none,    % Sort the references numbering by first appearance
  style=numeric,   % use numbers for references
]{biblatex}
\addbibresource{references.bib}  % the bib file to use
\DefineBibliographyStrings{english}{andothers = {{et\,al\adddot}}}  % replace u.a. with et al.


% Last packages, do not change order or insert new packages after these ones
\usepackage[pdfusetitle, unicode, linkbordercolor=tugreen, citebordercolor=tugreen]{hyperref}
\usepackage{bookmark}
\usepackage[shortcuts]{extdash}

%------------------------------------------------------------------------------
%-------------------------    Angaben zur Arbeit   ----------------------------
%------------------------------------------------------------------------------

\author{Nico Guth}
\title{Development of a multivariate algorithm for the classification of B~mesons at the LHCb experiment.}
\date{2022}
\birthplace{Dinslaken}
\chair{Arbeitsgruppe Albrecht}
\division{Fakultät Physik}
\thesisclass{Bachelor of Science}
\submissiondate{14.06.2022}
\firstcorrector{Prof.~Dr.~Johannes~Albrecht}
\secondcorrector{PD~Dr.~Dominik~Elsässer}

% tu logo on top of the titlepage
\titlehead{\includegraphics[height=1.5cm]{logos/tu-logo.pdf}}

\begin{document}
\begin{linenumbers}
\frontmatter
\maketitle

% Gutachterseite
\makecorrectorpage

% hier beginnt der Vorspann, nummeriert in römischen Zahlen
\thispagestyle{plain}

\section*{Abstract} %TODO
The precise measurements needed to further test the Standard Model of particle physics produce large amounts of data.
To analyse this data, machine learning algorithms have gained popularity over the last two decades due to their high efficiency and generalizability.
A classification algorithm is developed in this thesis for the distinction between uncharged $B_d$ and $B_s$ mesons in $pp$-collisions.
This algorithm uses data of tracks associated with the signal $B$ meson without data of the signal decay. 
The motivation behind developing this algorithm is a potential use case in other studies where some types of background of $B_d$ or $B_s$ mesons can be tough to identify using the signal decay kinematics.
The machine learning models used in this thesis are trained on simulated data of the LHCb detector. %TODO: of or on?
The chosen strategy is to first identify \enquote{same side} tracks using a Boosted Decision Tree classifier and then use the output of this classifier in a DeepSet classifier to identify the type of signal $B$ meson.
Evaluating this algorithm on simulated data shows a clear achieved separation between both $B$ meson types.
However, tests on LHCb data yield a substantially smaller separation.
These tests are performed using fits of the $B$ mass distribution after various selections based on the output of the DeepSet.

\section*{Kurzfassung (to be done)} %TODO!!
\begin{foreignlanguage}{german}
    The precise measurements needed to further test the Standard Model of particle physics produce large amounts of data.
    To analyse this data, machine learning algorithms have gained popularity over the last two decades due to their high efficiency and generalizability.
    A classification algorithm is developed in this thesis for the distinction between uncharged $B_d$ and $B_s$ mesons in $pp$-collisions.
    This algorithm uses data of tracks associated with the signal $B$ meson without data of the signal decay. 
    The motivation behind developing this algorithm is a potential use case in other studies where some types of background of $B_d$ or $B_s$ mesons can be tough to identify using the signal decay kinematics.
    The machine learning models used in this thesis are trained on simulated data of the LHCb detector. %TODO: of or on?
    The chosen strategy is to first identify \enquote{same side} tracks using a Boosted Decision Tree classifier and then use the output of this classifier in a DeepSet classifier to identify the type of signal $B$ meson.
    Evaluating this algorithm on simulated data shows a clear achieved separation between both $B$ meson types.
    However, tests on LHCb data yield a substantially smaller separation.
    These tests are performed using fits of the $B$ mass distribution after various selections based on the output of the DeepSet.
\end{foreignlanguage}

\tableofcontents

\mainmatter
% Hier beginnt der Inhalt mit Seite 1 in arabischen Ziffern
\chapter{Introduction}

Over the last centuries, experiments and techniques to probe physics on microscopic scales have become more and more precise.
Today, the highest precision is achieved at particle colliders such as the Large Hadron Collider (LHC).
Studying the production and decay processes of subatomic particles in high energetic particle collisions is a way to test our current understanding of particle physics.
The Standard Model of particle physics has been proven to make accurate predictions on almost all observed subatomic processes.
However, there are multiple physics phenomena this theory cannot explain.
This includes the existence of dark matter and dark energy, the imbalance between matter and antimatter in the universe and even gravity.
Therefore, further testing is still needed to find physics beyond the Standard Model.

The particle collisions themselves cannot be detected directly.
Reconstructing the proton-proton collisions at the LHC is made possible by using particle detectors such as the LHCb detector.
Here, the products of a collision are analysed to identify the particles as well as the decays of these particles.
Because of the complexity of this problem and the amount of data that needs to be analysed, in the last two decades more and more machine learning algorithms are used to support such measurements.

The goal of this thesis is to develop and train such an algorithm and evaluate its performance on LHCb data.
The purpose of this algorithm is to distinguish $B_d$ from $B_s$ mesons in proton-proton collisions based on hadronization and fragmentation tracks. The $B_d$ mesons include $B^0$ and $\bar{B}^0$ mesons and the $B_s$ mesons include $B_s^0$ and $\bar{B}_s^0$ mesons.




















% unintentionally too close to other ba theses:
%All the current physics research in particle physics is based on a construct of theories known as the Standard Model (SM). 
%Although it is currently one of the best tested theories in physics, scientists are certain that it is incomplete.
%The existence of dark matter and dark energy, and the imbalance between matter and antimatter in our universe cannot be explained using the %Standard Model.
%Additionally, the Theory of General Relativity is incompatible with the Standard Model.
%These are only a few of the reasons to further test the Standard Model. 
%In order to do this, very precise experiments are needed ... 

\chapter{Theoretical basics}

% TODO: Short intro to the chapter

\section{The Standard Model of particle physics}

To classify particles with different properties and quantify the interactions between those particles, physicists developed a construct of theories known as the Standard Model of particle physics.
It is based on quantum field theory which itself is the generalization of quantum mechanics to include special relativity and classical field theory. 




\chapter{Experimental foundation}

% TODO: Short introduction to the chapter

\section{The LHC}

The LHC (Large Hadron Collider) is a $\qty{27}{\km}$ long ring accelerator located at the border of France and Switzerland. 
It is the largest particle accelerator in the world, and it is managed by CERN (European Organization for Nuclear Research) in Geneva.
In it, protons can collide with a center-of-mass energy of $\qty{14}{\TeV}$ at a peak luminosity (interactions per area and time) of $\qty{e34}{\cm\squared\s}$ \cite{LHC}.
Additionally, the LHC can also accelerate heavy ions.

The four largest detectors at the LHC are ATLAS\cite{ATLAS}, CMS\cite{CMS}, ALICE\cite{ALICE} and LHCb\cite{LHCb}.
While ATLAS and CMS are general-purpose particle detectors, ALICE is dedicated to heavy ion collisions resulting in a quark-gluon plasma, and LHCb is dedicated to $pp$~collisions involving $c$ or $b$ quarks.

\section{The LHCb detector}

The LHCb detector is a particle detector with the main focus on decays in which $b$ or $c$ quarks are involved.
Its goals are precision measurements on the CP-violation and on rare decays. 
To accomplish this, it has been build as a single arm forward-spectrometer covering angles to the beam pipe from $\qty{10}{\milli\radian}$ to $\qty{300}{\milli\radian}$ \cite{LHCb}. 
This design has been chosen because the majority of high energetic $b$-/$c$-hadron pairs, produced in $pp$~collisions, have velocities in roughly the same direction as one of the incident protons.
An overview of the detector is shown in \autoref{fig:lhcb_detector}.
The following paragraphs briefly describe the components of the LHCb detector based on the article \enquote{The LHCb Detector at the LHC}\cite{LHCb}.

\begin{figure}
    \centering
    \includegraphics[width=\textwidth]{images/lhcb_detector.png}
    \caption{View of the LHCb detector \cite{LHCb}. }
    \label{fig:lhcb_detector}
\end{figure}

A particle produced at the primary vertex first traverses the Vertex Locator (VELO). 
There, the tracks of charged particles are measured using silicon semiconductor sensors at a resolution of about $\qty{4}{\micro\meter}$. 
This information then is primarily used to reconstruct the position of the primary and secondary vertices of the decay particles.

Beyond the VELO, the Tracking System then further measures the position over time of the same particles. 
This is done in the Tracker Turicensis (TT) before the magnet and in the Trackers T1, T2 and T3 behind the magnet.
The dipole magnet bends the particle tracks through the Lorentz force so that the particles' momentum can be inferred.
While the TT and the inner parts of the T1-T3 also use silicon semiconductor sensors, the outer parts of the T1-T3 use straw-tube modules which are gaseous ionization detectors. 

The Ring Imaging Cherenkov detectors (RICH1, RICH2) are used to measure the particle velocities.
By traversing optically dense materials, charged particles can induce the emission of Cherenkov radiation.
This radiation is then detected by photo multipliers. 
The opening angle of the circular emitted radiation is dependent on the velocity of the particle. 
The use of aerogel, $\symup{C}_4\symup{F}_{10}$ and $\symup{C}\symup{F}_4$ as radiators allows measuring particles with momenta between $\qty{1}{\GeV}$ and $\qty{100}{\GeV}$.
Comparing the momentum measured in the Tracking System with the velocity measured in the RICH detectors yields information about the particle identity based on its mass. 

The calorimeters (SPD/PS, ECAL, HCAL) stop most particles and measure their energy.
In dense materials particles induce particle showers with the size directly related to the energy deposited.
Each calorimeter is structured in alternating layers of absorbers and scintillators. 
The electronic calorimeter~(ECAL) absorbs electrons, positrons and photons using lead, and the hadronic calorimeter~(HCAL) absorbs any hadrons using iron.
The pad/preshower detector~(SPD/PS) is used to reject $\pi$~background detected in the ECAL.
The calorimeters are the only detectors at LHCb which can detect chargeless particles, and they are critical for particle identification.

The muon chambers (M1-M5) stop and track charged particles which pass through the calorimeters without absorption.
The only charged particles at this point in the detector are muons.
Iron layers are used to stop the muons and multi-wire proportional chambers  are used to detect them.

The only known particles that pass through the LHCb detector undetected are neutrinos.
They are indirectly measured through analysis of the missing transversal momentum and energy. 

A LHCb component not listed in \autoref{fig:lhcb_detector} is the Trigger System.
This system is needed to reduce the $\qty{40}{\MHz}$ beam crossing rate of the LHC to the $\qty{2}{\kHz}$ event detection rate of the LHCb. 
It consists of a hardware trigger (L0) and a software trigger (HLT).
The L0 collects information about the number of primary vertices from the pile-up system of the VELO.
It also attempts to reconstruct the highest $E_\text{T}$ in the calorimeters and the two highest $p_\text{T}$ in the muon chambers. 
This information is used to reject events until a detector read out rate of $\qty{1}{\MHz}$ is achieved.
The HLT then further triggers events asynchronously to this rate. 
For this, all detector outputs are evaluated for the involvement of $b$ hadrons in the events.
At the final rate of about $\qty{2}{\kHz}$, the event data is permanently written to hard drives and can be accessed on demand by all members of the LHCb collaboration.



\chapter{Methods and results}

% Belongs in the methodology chapter:
% The aim of this thesis is to distinguish $B_d$ and $B_s$ mesons without information about the signal decay.
% This means it is irrelevant which quarks are antiquarks, and it is expected that only information about the same side is relevant here.

\chapter{Conclusion}

In this thesis, a classification algorithm to distinguish $B_d$ from $B_s$ mesons in $pp$-collisions is developed, trained and evaluated.
This algorithm is intended to be independent of the signal $B$ decay channel and uses only data of the tracks associated with the signal $B$ meson without data of the signal decay.
For the training of supervised machine learning algorithms, simulated samples of the decays $B_d \rightarrow J/\psi \, K^*$ and $B_s \rightarrow D^+_s \, \pi^-$ are used in this thesis. This data is based on simulations of the LHCb detector in the year 2016.

The developed algorithm first uses a Boosted Decision Tree (BDT) classifier to identify \enquote{same side} (SS) fragmentation tracks.
The SS fragmentation contains particles that are related to the $d\bar{d}$- or $s\bar{s}$-pair that is part of the production of the signal $B$. 
In contrast to other tracks in the data these SS tracks therefore contain relevant information for the distinction between $B_d$ and $B_s$ mesons.
The output of this BDT is then used in a DeepSet classifier for the signal $B$.

A BDT is trained using all available features and features are then selected based on feature importance metrics called the gain and the permutation importance of the ROC AUC. 
The final BDT uses 21 features, and a ROC AUC of $0.763$ is achieved on test samples.
%After training the final BDT with a reduced feature set, the achieved separation between SS tracks and other tracks is evaluated using a ROC curve.
%A ROC AUC of $0.763$ is achieved on test data.
Although the achieved separation is not sufficient to correctly identify most of the SS tracks, this BDT output can potentially help the DeepSet to form a decision.

For the feature selection of the DeepSet, the permutation importances of the accuracy and the ROC AUC are used.
This leads to the 23 features listed in \cref{tab:B_features} that are used in the final DeepSet for the signal $B$ classification.
An indication that the BDT output ($\text{Prob}_\text{SS}$) has an impact on the DeepSet performance can be seen in \cref{fig:B_importances}.
This however has to be interpreted with care, because feature importances can only be seen as estimates.
The trained DeepSet achieves a ROC AUC of $0.739$ on test data.
The ROC curve and the distribution of the DeepSet output are shown in \cref{fig:B_eval}.

It can be seen that a clear separation of the $B$ mesons is achieved on simulation.
However, any algorithm that has been developed with simulated samples has to be tested on data that originates from actual measurements. 
Therefore, the developed algorithm of this thesis is tested on LHCb data of the decays $B_d \text{ or } B_s \rightarrow J/\psi \, K^0_\text{S}$.

Compared to the training data, LHCb data contains background events that do not actually involve $B_d$ or $B_s$ mesons.
Since evaluating the $B$ classification requires clearly visible $B_s$ peaks in the $B$ mass distribution, the number of background events has to be reduced beforehand.
This background reduction is done using a combination of manual cuts and a BDT that is trained on simulated samples of the decay $B_d \rightarrow J/\psi \, K^0_\text{S}$.
This BDT achieves a ROC AUC of $0.989$ on test data.
The ROC curve achieved by the BDT and the total background reduction is shown in \cref{fig:BKG_eval}.

The developed algorithm of this thesis is then applied to the background reduced data.
To evaluate the achieved separation, $B$ mass distributions with various selection criteria based on the DeepSet output ($\text{Prob}_{B_s}$) are used to fit a function.
The fit components that represent the combinatorial background, the $B_d$ peak and the $B_s$ peak are then integrated numerically.
This results in estimated counts of events attributed to the $B_d$ and $B_s$ mesons remaining after each selection.

The resulting ratios of the number of $B_s$ meson events over the number of $B_d$ meson events are shown in \cref{fig:data_ratio}.
Here, selection criteria of the form $\text{Prob}_{B_s} \geq x$ show an upward trend with increasing $x$ values.
This could be an indication of some level of achieved separation between $B_s$ and $B_d$ mesons.
Selection criteria of the form $\text{Prob}_{B_s} \leq x$ show an increase of this ratio for lower $x$ values.
However, for small values of $x$ in $\text{Prob}_{B_s} \leq x$ there are too many combinatorial background events and an estimation of the $B_s$ peak is almost impossible. 
This is not the case for large values of $x$ in $\text{Prob}_{B_s} \geq x$ where a clear $B_s$ peak is visible.
All calculated ratios are compatible with the expected ratio for no achieved separation, but the upward trend for $\text{Prob}_{B_s} \geq x$ selections suggests at least a small achieved separation.

Another method to test the achieved separation is to plot the efficiencies of both the $B_d$ and $B_s$ mesons similar to a ROC curve.
This is shown in \cref{fig:data_roc}.
Here, no clear signs of an achieved separation are visible.

Overall, the classification algorithm for separating $B_s$ and $B_d$ mesons was designed and tested on simulation and LHCb data. 
Unfortunately, the successful separation of $B_s$ and $B_d$ mesons on simulated samples could not be reproduced on LHCb data.
However, the achieved separation on simulation is an indication that the strategy to use a DeepSet to distinguish $B_d$ from $B_s$ mesons based on data of the associated event could succeed. 
This raises the question whether the classifier exploited selection differences in the training datasets that went unnoticed, or if a simulated feature was mismodeled. 
In a future analysis it will therefore be important to show that observed kinematic differences in the training simulation are only due to differences in the $B_d$ and $B_s$ masses and that their distributions closely match the data. Since it cannot be decided with absolute certainty that kinematic differences in the training simulation are only due to $\Bd$ vs $\Bs$ mass differences and not due to indirect selection constraints, an attempt could be made at equalizing the kinematics in the training data through reweighting or similar methods.


\appendix
% Hier beginnt der Anhang, nummeriert in lateinischen Buchstaben
\input{content/a_appendix.tex}

\backmatter
\printbibliography

\cleardoublepage
% From https://www.tu-dortmund.de/studierende/im-studium/pruefungsangelegenheiten/allgemeine-vordrucke/
\includepdf{content/Eidesstattliche_Versicherung.pdf}

\end{linenumbers}
\end{document}
