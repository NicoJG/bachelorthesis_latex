\section{$B$ meson production in proton-proton collisions}

A proton-proton collision ($pp$-collision) is the interaction between two high energetic protons through the strong force.
This chaotic process produces many particles.





A proton-proton collision ($pp$-collision) is the interaction between two high energetic protons through the strong force.
This chaotic process is often initiated by the exchange of two gluons fusing into one gluon.
The decay of this gluon into two quarks starts a process called hadronization.
In hadronization, quark pairs emerge from the vacuum to combine with existing quarks into hadrons.




When this gluon decays into a $b\bar{b}$ pair, a process called hadronization starts.
In hadronization, quark pairs emerge from the vacuum to combine with existing quarks into hadrons. 


One of those quarks may combine with a $b$ quark from the gluon-gluon fusion to produce a $B$ meson.
$B$ mesons are composed of one $b$ quark and one of the $u/d/c/s$ quarks.
Relevant for this thesis are the uncharged $B$ mesons
\begin{align*}
    B^0 &= \bar{b}d \, , & \bar{B}^0 &= b\bar{d} \, , & B_s^0 &= \bar{b}s \, , & \bar{B}_s^0 &= b\bar{s} \, .
\end{align*}

In a proton-proton collision ($pp$-collision) the exchange of gluons initiates the production of many particles.

When two high energetic protons approach each other, they may interact through the exchange of gluons.
Those gluons may fuse





In a proton-proton collision ($pp$-collision) many different particles are produced through the strong interaction.
When this includes the decay of a gluon into a $b\bar{b}$ pair, it can lead to the production of a $B$ meson.
$B$ mesons are composed of one $b$ quark and one of the $u/d/c/s$ quarks.
Relevant for this thesis are the uncharged $B$ mesons
\begin{align*}
    B^0 &= \bar{b}d \, , & \bar{B}^0 &= b\bar{d} \, , & B_s^0 &= \bar{b}s \, , & \bar{B}_s^0 &= b\bar{s} \, .
\end{align*}

At LHCb, a frequent goal is to analyse $B$ meson candidates in $pp$ collisions, often called the signal $B$.
This can be done by using information about the particles of the signal decay. 
Or it can be done by analysing all particles which are associated to the signal $B$.
The latter method is independent of the signal decay channel, and it is often used in flavour tagging algorithms at LHCb. 
A similar method is used in this thesis.
An example sketch of such an associated event is shown in \autoref{fig:ft_scheme}. 
This sketch excludes particles which are associated with the $pp$ collision event but not with the signal $B$.
Those particles are called the background of an event.

\begin{figure}
    \centering
    \includegraphics[width=\textwidth]{images/FlavourTaggingScheme.pdf}
    \caption{Schematic overview of the underlying principles of LHCb's flavour tagging algorithms \cite{ft_scheme}.}
    \label{fig:ft_scheme}
\end{figure}

The collision point of a $pp$ pair is called the primary vertex~(PV), and the decay point of the signal $B$ is called the secondary vertex~(SV).
Due to the high mass of $b$ quarks, an associated event can roughly be separated into two sides.
The same side~(SS) contains particles of the signal decay.
It also contains particles produced in association with the non-$b$ valence quark of the $B$ meson.
These particles are called the SS fragmentation.
The opposite side~(OS) contains all particles associated with the $b$ quark not in the signal $B$.
The combined information about the SS fragmentation and the OS allows the identification of the signal $B$ flavour.
