\section{$B$ mesons in proton-proton collisions}

$B$ mesons are composed of one $b$ quark and one of the $u/d/c/s$ quarks.
Relevant for this thesis are the uncharged $B$ mesons
\begin{align*}
    B^0 &= \bar{b}d \, , & \bar{B}^0 &= b\bar{d} \, , & B_s^0 &= \bar{b}s \, , & \bar{B}_s^0 &= b\bar{s} \, .
\end{align*}

In a proton-proton collision many different particles can be produced through the strong force.
The origin of a $B$ meson is the production of a $b\bar{b}$ pair from a gluon.
This $B$ meson then decays into other particles which can be detected.
These particles are the signal decay particles.
Since the $b$ quark has to pair up with a different quark which itself is produced 

For a $B$ meson to be produced it is necessary that a $b\bar{b}$-pair is produced

In a $pp$-collision $B$-mesons are produced in pairs, because of the production of $b\bar{b}$-pairs through the strong interaction. 
The $B$-meson of interest for an analysis is called the signal $B$ and its partner is called the opposite $B$.
The underlying event is therefore often divided into same side tracks and opposite side tracks, depending on the origin of its particle.
Each side can then be further divided into fragmentation tracks and decay tracks. (see \autoref{fig:ft_scheme})
At LHCb, this track classification is often used in flavour tagging algorithms, and it is also used in this thesis.

\begin{figure}
    \centering
    \includegraphics[width=\textwidth]{images/FlavourTaggingScheme.pdf}
    \caption{Track classification used in flavour tagging algorithms. \cite{ft_scheme}}
    \label{fig:ft_scheme}
\end{figure}


%\begin{table}
%    \centering
%    \caption{Listing of $B$-mesons and their quark contents.}
%    \begin{tabular}{c c c c | c c c c}
%        \toprule
%        \multicolumn{4}{c}{uncharged $B$-mesons} & \multicolumn{4}{c}{charged $B$-mesons} \\
%        \midrule
%        $B_d^0$ & $\bar{B}_d^0$ & $B_s^0$ & $\bar{B}_s^0$ & $B_u^-$ & $B_u^+$ & $B_c^-$ & $B_c^+$ \\
%        $b\bar{d}$ & $\bar{b}d$ & $b\bar{s}$ & $\bar{b}s$ & $b\bar{u}$ & $\bar{b}u$ & $b\bar{c}$ & $\bar{b}c$ \\
%        \bottomrule
%    \end{tabular}
%    \label{tab:B_mesons}
%\end{table}

%\begin{table}
%    \centering
%    \caption{Listing of $B$-mesons and their quark contents.}
%    \begin{tabular}{c c | c c}
%        \toprule
%        \multicolumn{2}{c}{uncharged $B$-mesons} & \multicolumn{2}{c}{charged $B$-mesons} \\
%        \midrule
%        $B^0 = b\bar{d}$ & $B_s^0 = b\bar{s}$ & $B^- = b\bar{u}$ & $B_c^- = b\bar{c}$ \\
%        $\bar{B}^0 = \bar{b}d$ & $\bar{B}_s^0 = \bar{b}s$ & $B^+ = \bar{b}u$ & $B_c^+ = \bar{b}c$ \\
%        \bottomrule
%    \end{tabular}
%    \label{tab:B_mesons}
%\end{table}

%\begin{align*}
%    \text{uncharged:} & B_d^0 = b\bar{d}, & \bar{B}_d^0 = \bar{b}d, & B_s^0 = b\bar{s}, & \bar{B}_s^0 = \bar{b}s \\
%    \text{charged:} & B_u^- = b\bar{u} & B_u^+ = \bar{b}u & B_c^- = b\bar{c} & B_c^+ = \bar{b}c
%\end{align*}



%The decay channels used in this thesis are
%\begin{align*}
%    B_d^0/\bar{B}_d^0 &\rightarrow J/\psi + K*^0 \, , \\
%    B_s^0/\bar{B}_s^0 &\rightarrow D_s^- + \pi^+ \, \text{and} \\
%    B_d^0/\bar{B}_d^0/B_s^0/\bar{B}_s^0 &\rightarrow J/\psi + K_S^0 \, .
%\end{align*}