\section{The Standard Model of particle physics}

% SM intro
Physics phenomena on the smallest observed length scales are described by a quantum field theory known as the Standard Model~(SM).
The SM contains all known elementary particles and the interactions between them.
The elementary particles are separated into twelve fermions with spin~$\frac{1}{2}$, four gauge bosons with spin $1$ and the Higgs boson with spin~$0$.
Each of those particles has a corresponding antiparticle with all their charge-like properties inverted.
Particles can be identical to their antiparticles.
While fermions make up all the currently observable matter in the universe, bosons describe the interactions and properties of every particle.
The following paragraphs briefly introduce each fundamental particle listed in \autoref{tab:sm_particles}.

\begin{table}
    \centering
    \caption{Listing of all fundamental particles in the Standard Model.}
    \begin{tabular}{c c c | c c c | c c c c | c}
        \toprule
        \multicolumn{6}{c|}{fermion} & \multicolumn{5}{c}{boson} \\
        \multicolumn{3}{c}{quark} & \multicolumn{3}{c|}{lepton} & \multicolumn{4}{c}{gauge-boson} & \multicolumn{1}{c}{scalar-boson} \\
        \midrule
        $u$ & $c$ & $t$ & $e^\pm$ & $\mu^\pm$ & $\tau^\pm$ & \multirow{2}{*}{$\gamma$} & \multirow{2}{*}{$g$} & \multirow{2}{*}{$Z$} & \multirow{2}{*}{$W^\pm$} & \multirow{2}{*}{$H^0$} \\
        $d$ & $s$ & $b$ & $\nu_e$ & $\nu_\mu$ & $\nu_\tau$ &&&&& \\
        \bottomrule
    \end{tabular}
    \label{tab:sm_particles}
\end{table}

% fermions
The atomic matter in the universe consists of mainly three of the elementary fermions. 
These are the up quark with charge~$\frac{2}{3}e$, the down quark with charge~$-\frac{1}{3}e$ and the electron with charge~$-1e$.
The up and down quarks are the valence quarks of the protons and neutrons inside atomic nuclei.
Additionally, processes of the weak interaction often involve the neutrally charged electron-neutrino.
These four fermions, together with their antiparticle, make up the so-called first particle generation.
The second and third particle generations have the same structure as the first generation, but the masses of the corresponding particles increase with each generation.
Each generation has one up-type quark (up~$u$, charm~$c$, top~$t$), one down-type quark (down~$d$, strange~$s$, bottom~$b$), one charged lepton (electron~$e$, muon~$\mu$, tau~$\tau$) and one uncharged lepton (electron-neutrino~$\nu_e$, muon-neutrino~$\nu_\mu$, tau-neutrino~$\nu_\tau$).
Originally, neutrinos were assumed to be massless in the SM, but this changed with the discovery of neutrino oscillations \cite{NeutrinoMassSK,NeutrinoMassSNO}. 

% bosons
The interactions between all particles are described by the four fundamental forces (electromagnetism, the strong force, the weak force, gravity).
While the SM does not include gravity, the gauge bosons are the mediators of the three other forces.

The photon~$\gamma$ is charge- and massless, and it mediates the electromagnetic force.
The electromagnetic force is described in quantum electrodynamics~(QED) and couples to the charge $Q$ of a particle.

The gluon~$g$ is also charge- and massless, and it mediates the strong force.
The strong force is described in quantum chromodynamics~(QCD) and couples to the color-charge (red, green, blue, anti-red, anti-green, anti-blue) of a particle.
Only quarks and gluons are color-charged.
This means, in contrast to QED, the force-carrier particles in QCD can interact with themselves.
Also, the force between two color-charged particles increases with their distance.
This leads to the so-called color-confinement, which states that only neutrally color-charged particles can exist in isolation.
A particle is color neutral if either a color is canceled with the corresponding anti-color or all three colors are present in equal amounts.
This prohibits the direct observation of quarks, and only collections of quarks, called hadrons, are observable.
There are multiple classes of hadrons, but the most common classes are mesons~(one quark and one antiquark) and baryons~(three quarks).
Tetraquarks and pentaquarks were first observed at LHCb in 2015 and 2016 \cite{TetraquarkLHCb,PentaquarkLHCb}. 
The bound state of neutrons~($udd$) and protons~($uud$) in the atomic nucleus is also the result of the strong force.

The mediators of the weak force are the chargeless $Z$-boson and the $W^\pm$-boson with charge $\pm 1e$.
The $W^\pm$ boson couples to the third component of the weak isospin $T_3 = \pm\frac{1}{2}$.
The $Z$ boson couples to the weak hypercharge $Y_W = 2(Q - T_3)$.
Charged currents through the $W^\pm$ boson are the only known way to change the flavour of a particle.
There are six quark flavours and six lepton flavours matching the types of fermions listed in \autoref{tab:sm_particles}.
This results in processes such as the $\beta^-$-decay ($n \rightarrow p + e^- + \bar{\nu}_e$) or the $\beta^+$-decay ($p \rightarrow n + e^+ + \nu_e$).

The Higgs boson $H^0$ is chargeless, and it is an excitation of the Higgs field.
Through the interaction with the Higgs field all fermions as well as the $W^\pm$ and $Z$ bosons gain their masses.
%The Higgs field interacts with all fermions as well as the $W^\pm$ and $Z$ bosons, and it gives particles their mass.
The Higgs boson is the last experimentally found elementary particle and was first observed at ATLAS and at CMS in 2012 \cite{HiggsATLAS,HiggsCMS}. 
