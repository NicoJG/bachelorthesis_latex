\section{The Standard Model of particle physics}

% SM intro
Physics phenomena on the smallest observed length scales are described by a quantum field theory known as the Standard Model (SM).
The SM contains all known elementary particles and the interactions between them.
A particle is called a boson if it has an integer spin, and it is called a fermion if it has a half integer spin.
There are twelve fermions with spin $\frac{1}{2}$, four gauge bosons with spin $1$ and the Higgs boson with spin $0$.
Each of those particles has a corresponding antiparticle with all their charge-like properties inverted, but all particles with no charge-like property are identical to their antiparticle.
The fermions make up all the currently observable matter in the universe.
The gauge bosons describe the interactions between particles.
And the Higgs boson describes how particles get their masses.
The following paragraphs briefly introduce each fundamental particle listed in \autoref{tab:sm_particles}.

\begin{table}
    \centering
    \caption{Listing of all fundamental particles in the Standard Model.}
    \begin{tabular}{c c c | c c c | c c c c | c}
        \toprule
        \multicolumn{6}{c|}{fermion} & \multicolumn{5}{c}{boson} \\
        \multicolumn{3}{c}{quark} & \multicolumn{3}{c|}{lepton} & \multicolumn{4}{c}{gauge-boson} & \multicolumn{1}{c}{scalar-boson} \\
        \midrule
        $u$ & $c$ & $t$ & $e^\pm$ & $\mu^\pm$ & $\tau^\pm$ & \multirow{2}{*}{$\gamma$} & \multirow{2}{*}{$g$} & \multirow{2}{*}{$Z$} & \multirow{2}{*}{$W^\pm$} & \multirow{2}{*}{$H^0$} \\
        $d$ & $s$ & $b$ & $\nu_e$ & $\nu_\mu$ & $\nu_\tau$ &&&&& \\
        \bottomrule
    \end{tabular}
    \label{tab:sm_particles}
\end{table}

% fermions
The atomic matter in the universe consists of mainly three of the elementary fermions. 
These are the up quark with charge $\frac{2}{3}e$, the down quark with charge $-\frac{1}{3}e$ and the electron with charge $-1e$.
The up and down quarks are the valence quarks of the protons and neutrons inside atomic nuclei.
Additionally, in processes involving the weak force the neutrally charged electron-neutrino is often produced.
These four fermions, together with their antiparticle, make up the so-called first particle generation.
The second and third particle generations have the same structure as the first generation, but the masses of the corresponding particles increase with each generation.
Each generation has one up-type quark (up~$u$, charm~$c$, top~$t$), one down-type quark (down~$d$, strange~$s$, bottom~$b$), one charged lepton (electron~$e$, muon~$\mu$, tau~$\tau$) and one uncharged lepton (electron-neutrino~$\nu_e$, muon-neutrino~$\nu_\mu$, tau-neutrino~$\nu_\tau$).
Originally, neutrinos were assumed to be massless in the SM, but this changed with the discovery of neutrino oscillations. \cite{NeutrinoMassSK,NeutrinoMassSNO}

% bosons
The interactions between all particles are described by the four fundamental forces electromagnetism, the strong force, the weak force and gravity.
While the SM does not include gravity, the gauge bosons are the mediators of the three other forces.

The photon~$\gamma$ is chargeless and mediates the electromagnetic force.
The electromagnetic force is described in quantum electrodynamics (QED) and couples to the charge of a particle.

The gluon~$g$ mediates the strong force.
The strong force is described in quantum chromodynamics (QCD) and couples to the color-charge (red, green, blue, anti-red, anti-green, anti-blue) of a particle.
In contrast to QED, not only the quarks are color-charged but also the force-carrier particle, the gluon, is color-charged.
This leads to the so-called color-confinement, which states that only neutrally color-charged particles can exist in isolation.
A particle is color neutral if either a color is canceled with the corresponding anti-color or all three colors are present in equal amounts.
This is why quarks cannot be observed directly, but only collections of quarks called hadrons.
There are multiple classes of hadrons, but the most common classes are mesons (one quark and one antiquark) and baryons (three quarks).
Tetraquarks (four quarks) and pentaquarks (five quarks) were first observed at LHCb in 2015 and 2016 \cite{TetraquarkLHCb,PentaquarkLHCb}. 
The strong force is also responsible for binding of neutrons ($udd$) and protons ($uud$) in the atomic nucleus.

The mediators of the weak force are the chargeless $Z$-boson and the $W^\pm$-boson with charge $\pm 1e$.
The weak force couples to the third component of the weak isospin ($+\frac{1}{2}$ for up-type quarks and charged leptons, $-\frac{1}{2}$ for down-type quarks and neutrino).
It is the only known way to change the flavour of a particle.
There are six quark flavours and six lepton flavours matching the types of fermions listed in \autoref{tab:sm_particles}.
The $Z$-boson couples to every other particle in the Standard Model except of the gluon.
The $W$-boson couples any charged lepton to the corresponding neutrino and any up-type quark to any down-type quark.
This results in processes such as the $\beta^-$-decay ($n \rightarrow p + e^- + \bar{\nu}_e$) or the $\beta^+$-decay ($p \rightarrow n + e^+ + \nu_e$).

The Higgs-boson~$H$, together with the Higgs mechanism and the Higgs field, is responsible for the masses of each particle.
It is chargeless, has spin $0$ and is part of the Higgs mechanism, in which the interaction of a particle with the Higgs field results in its mass.
It is the last experimentally found fundamental particle and was first observed at ATLAS and at CMS in 2012. \cite{HiggsATLAS,HiggsCMS}
