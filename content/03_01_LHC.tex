\section{The LHC}

The LHC (Large Hadron Collider) is a $\qty{27}{\km}$ long ring accelerator located at the border of France and Switzerland. 
It is the largest particle accelerator in the world, and it is managed by CERN (European Organization for Nuclear Research) in Geneva.
In it, protons can collide with a center-of-mass energy of $\qty{14}{\TeV}$ at a peak luminosity (interactions per area and time) of $\qty{e34}{\cm\squared\s}$ \cite{LHC}.
Additionally, the LHC can also accelerate heavy ions.

The four largest detectors at the LHC are ATLAS\cite{ATLAS}, CMS\cite{CMS}, ALICE\cite{ALICE} and LHCb\cite{LHCb}.
While ATLAS and CMS are general-purpose particle detectors, ALICE is dedicated to heavy ion collisions resulting in a quark-gluon plasma, and LHCb is dedicated to $pp$~collisions involving $c$ or $b$ quarks.
