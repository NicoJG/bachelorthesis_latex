\thispagestyle{plain}

\section*{Abstract}
The precise measurements needed to further test the Standard Model of particle physics produce large amounts of data.
To analyse this data, machine learning algorithms have gained popularity over the last two decades due to their high efficiency and generalizability.
In particular, machine learning algorithms can help separate signal from background in studies where some types of background of neutral $B_d$ or $B_s$ mesons can be tough to identify using the signal decay kinematics alone.
A classification algorithm is developed in this thesis for the distinction between neutral $B_d$ and $B_s$ mesons in $pp$-collisions.
This algorithm uses data of tracks associated with the signal $B$ meson without data of the signal decay. 
The machine learning models used in this thesis are trained on simulated data. 
The chosen strategy is to first identify signal side fragmentation tracks using a Boosted Decision Tree classifier and then use the output of this classifier in a DeepSet classifier to identify the type of signal $B$ meson.
Evaluating this algorithm on simulated data shows a clear achieved separation between both $B$ meson types.
However, tests on LHCb data yield a substantially smaller separation.

\section*{Kurzfassung (to be done)} %TODO!!
\begin{foreignlanguage}{german}
    Die präzisen Messungen, die nötig sind um das Standard Modell der Teilchenph
\end{foreignlanguage}
