\thispagestyle{plain}

\section*{Abstract} %TODO
The precise measurements needed to further test the Standard Model of particle physics produce large amounts of data.
To analyse this data, machine learning algorithms have gained popularity over the last two decades due to their high efficiency and generalizability.
A classification algorithm is developed in this thesis for the distinction between uncharged $B_d$ and $B_s$ mesons in $pp$-collisions.
This algorithm uses data of tracks associated with the signal $B$ meson without data of the signal decay. 
The motivation behind developing this algorithm is a potential use case in other studies where some types of background of $B_d$ or $B_s$ mesons can be tough to identify using the signal decay kinematics.
The machine learning models used in this thesis are trained on simulated data of the LHCb detector. %TODO: of or on?
The chosen strategy is to first identify \enquote{same side} tracks using a Boosted Decision Tree classifier and then use the output of this classifier in a DeepSet classifier to identify the type of signal $B$ meson.
Evaluating this algorithm on simulated data shows a clear achieved separation between both $B$ meson types.
However, tests on LHCb data yield a substantially smaller separation.
These tests are performed using fits of the $B$ mass distribution after various selections based on the output of the DeepSet.

\section*{Kurzfassung (to be done)} %TODO!!
\begin{foreignlanguage}{german}
    The precise measurements needed to further test the Standard Model of particle physics produce large amounts of data.
    To analyse this data, machine learning algorithms have gained popularity over the last two decades due to their high efficiency and generalizability.
    A classification algorithm is developed in this thesis for the distinction between uncharged $B_d$ and $B_s$ mesons in $pp$-collisions.
    This algorithm uses data of tracks associated with the signal $B$ meson without data of the signal decay. 
    The motivation behind developing this algorithm is a potential use case in other studies where some types of background of $B_d$ or $B_s$ mesons can be tough to identify using the signal decay kinematics.
    The machine learning models used in this thesis are trained on simulated data of the LHCb detector. %TODO: of or on?
    The chosen strategy is to first identify \enquote{same side} tracks using a Boosted Decision Tree classifier and then use the output of this classifier in a DeepSet classifier to identify the type of signal $B$ meson.
    Evaluating this algorithm on simulated data shows a clear achieved separation between both $B$ meson types.
    However, tests on LHCb data yield a substantially smaller separation.
    These tests are performed using fits of the $B$ mass distribution after various selections based on the output of the DeepSet.
\end{foreignlanguage}
