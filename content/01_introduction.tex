\chapter{Introduction}

Over the last centuries, experiments and techniques to probe physics on microscopic scales have become more and more precise.
Today, the highest precision is achieved at particle colliders such as the Large Hadron Collider (LHC).
Studying the production and decay processes of subatomic particles in high energetic particle collisions is a way to test our current understanding of particle physics.
The Standard Model of particle physics has been proven to make accurate predictions on almost all observed subatomic processes.
However, there are multiple physics phenomena this theory cannot explain.
This includes the existence of dark matter and dark energy, the imbalance between matter and antimatter in the universe and even gravity.
Therefore, further testing is still needed to find physics beyond the Standard Model.

The particle collisions themselves cannot be detected directly.
Reconstructing the proton-proton collisions at the LHC is made possible by using particle detectors such as the LHCb detector.
Here, the products of a collision are analysed to identify the particles as well as the decays of these particles.
Because of the complexity of this problem and the amount of data that needs to be analysed, in the last two decades more and more machine learning algorithms are used to support such measurements.

The goal of this thesis is to develop and train such an algorithm and evaluate its performance on LHCb data.
The purpose of this algorithm is to distinguish $B_d$ from $B_s$ mesons in proton-proton collisions based on hadronization and fragmentation tracks. 




















% unintentionally too close to other ba theses:
%All the current physics research in particle physics is based on a construct of theories known as the Standard Model (SM). 
%Although it is currently one of the best tested theories in physics, scientists are certain that it is incomplete.
%The existence of dark matter and dark energy, and the imbalance between matter and antimatter in our universe cannot be explained using the %Standard Model.
%Additionally, the Theory of General Relativity is incompatible with the Standard Model.
%These are only a few of the reasons to further test the Standard Model. 
%In order to do this, very precise experiments are needed ... 
