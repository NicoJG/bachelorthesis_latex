\section[Classification of \texorpdfstring{$B_d$ and $B_s$}{Bd and Bs} mesons]{Classification of \texorpdfstring{$\symbf{B_d}$ and $\symbf{B_s}$}{Bd and Bs} mesons}

The distinction between $B_d$ and $B_s$ mesons based on the associated event is a non-trivial problem.
Since both mesons have different masses and quark contents, there are difference in the kinematics and types of particles in the associated event.
Therefore, in this thesis it is tested whether a multivariate machine learning algorithm can spot these differences.
The algorithm of choice for this thesis is a DeepSet that is trained on simulated LHCb data.
As explained in \cref{sec:DeepSet}, a DeepSet can handle data of variable length where the order of its elements should not change the result of the classification.
Therefore, DeepSets are ideal to classify $pp$-collision events that contain different amounts of particle tracks.
The DeepSet trained here, outputs an estimated probability $\text{Prob}_{B_s}$ for each event that the signal~$B$ is a $B_s$ meson.
Therefore, $1-\text{Prob}_{B_s}$ is the estimated probability that the signal~$B$ is a $B_d$ meson.
The implementation of the DeepSet is done with the library \textsc{PyTorch} \cite{pytorch}.

The procedure of selecting important features is similar to the procedure described in \cref{sec:SS_classifier}.
However, a DeepSet was trained on all features and the feature importances used here are the permutation importance of the ROC AUC and the permutation importance of the accuracy.
Again, features with the lowest feature importance scores are discarded until a reasonable amount of features is achieved.
The remaining 23 features are listed in \cref{tab:B_features} and the corresponding feature importances are shown in \cref{fig:B_importances}.

\begin{figure}
    \centering
    \includegraphics[width=\textwidth]{images/B_feature_importances.pdf}
    \caption{Calculated feature importances on the trained DeepSet. Shown are the permutation importances of the ROC AUC and the accuracy normed on the largest value.}
    \label{fig:B_importances}
\end{figure}

\begin{table}
    \centering
    \caption{List of all features used to train the DeepSet for B meson classification.}
    \label{tab:B_features}
    \begin{tabular}{c c}
        \toprule
        feature & feature \\
        \midrule
        $p$                 & $\text{Prob}_\text{SS}$ \\ %"Tr_T_P","Tr_ProbSS"
        $p_\text{T}$        & $\text{Prob}_e$ \\ %"Tr_T_PT", "Tr_T_PROBNNe"
        $p_\text{proj}$     & $\text{Prob}_\text{ghost}$ \\ %"Tr_p_proj","Tr_T_PROBNNghost"
        $\Delta p$          & $\text{Prob}_K$ \\ %"Tr_diff_p", "Tr_T_PROBNNk"
        $\Delta p_\text{T}$ & $\text{Prob}_\mu$ \\ %"Tr_diff_pt","Tr_T_PROBNNmu"
        $\Delta z$          & $\text{Prob}_p$ \\ %"Tr_diff_z", "Tr_T_PROBNNp"
        $\cos(\Delta \phi)$ & $\text{Prob}_\pi$ \\ %"Tr_cos_diff_phi", "Tr_T_PROBNNpi"
        $\Delta \eta$       & $\sigma(\text{IP}_\text{pileup vtx})$ \\ %"Tr_diff_eta", "Tr_T_IP_trPUS"
        $\text{IP}_\text{SV}$        & $Q_\text{VELO}$ \\ %"Tr_T_IP_trMother",  "Tr_T_VeloCharge"
        $\chi^2(\text{IP}_\text{SV})$    & SumBDT \\ %"Tr_T_IPCHI2_trMother", "Tr_T_SumBDT_ult"
        $\text{IP}_\text{min}$               & MinBDT \\ %"Tr_T_MinIP", "Tr_T_MinBDT_ult"
        $\chi^2(\text{IP}_\text{min})$           & \\ %"Tr_T_MinIPChi2"
        \bottomrule
    \end{tabular}
\end{table}

%(Explain all features)....
Multiple features that are used here are already described in \cref{sec:SS_classifier}.
These features are 
$p$,
$p_\text{T}$, 
$p_\text{proj}$, 
$\Delta p_\text{T}$, 
$\Delta z$, 
$\cos(\Delta \phi)$, 
$\Delta \eta$, 
$\text{Prob}_\text{ghost}$
$\text{IP}_\text{SV}$, 
$\chi^2(\text{IP}_\text{SV})$, 
$\text{IP}_\text{min}$,        
SumBDT and
MinBDT.
The other features used here are described in the following paragraph.

All the following features describe a single particle or rather its reconstructed track.
$\Delta p$ is the difference of the absolute momenta of the particle and the signal~$B$.
$\chi^2(\text{IP}_\text{min})$ is the uncertainty of $\text{IP}_\text{min}$.
%
$\text{Prob}_\text{SS}$ is an estimated probability that the track belongs to the SS and is the output of the classifier described in \cref{sec:SS_classifier}. 
$\text{Prob}_e$, 
$\text{Prob}_K$, 
$\text{Prob}_\mu$, 
$\text{Prob}_p$ and 
$\text{Prob}_\pi$ are estimated probabilities about the particle's type.
%
$Q_\text{VELO}$ is a measure for the interaction of the particle with the VELO detector based on the induced current due to material interactions and bremsstrahlung photons.


The final DeepSet consists of a $\phi$-network that extracts latent space vectors of length 64 for each track and a $\rho$-network which transforms the sum of these vectors to a scalar output.
The $\phi$-network has an input layer of size 23 followed by 3 layers with of sizes 64, 128 and 64.
The $\rho$-network has an input layer of size 64 followed by 3 layers of sizes 128, 64 and 1.
All hidden layers use the ReLU activation function and have a dropout rate of $50\%$.
The output layer of the $\rho$-network uses the sigmoid activation function.

From the $0.4$ million events in the dataset, $60\%$ are used for training and $40\%$ are used for validation of the trained model.
Before training or evaluation, each feature in the data is rescaled using 
\begin{equation}
    f(x) = \frac{x - \mu_i}{\sigma_i} 
\end{equation} 
to ensure a common order of magnitude for all input features.
Where $\mu_i$ is the mean and $\sigma_i$ is the standard deviation of feature $i$ in the training data.

The DeepSet is trained using the loss function \enquote{binary cross entropy} and the optimizer \enquote{Adam}.
Using early stopping, the training is stopped at iteration 600 after 50 iterations without improvement on the validation dataset.
The loss and the error rate for each iteration of the training of the DeepSet are shown in \cref{fig:B_history}.

\begin{figure}
    \centering
    \begin{subfigure}{0.5\textwidth}
        \centering
        \includegraphics[width=\textwidth]{images/B_history_loss.pdf}
        \caption{binary cross entropy loss}
    \end{subfigure}%
    \begin{subfigure}{0.5\textwidth}
        \centering
        \includegraphics[width=\textwidth]{images/B_history_error.pdf}
        \caption{error rate}
    \end{subfigure}%
    \caption{Training performance of the DeepSet for $B$ meson classification.}
    \label{fig:B_history}
\end{figure}

To show the achieved separation of the $B$ mesons, histograms of $\text{Prob}_{B_s}$ split by the true labels of the simulation are shown in \cref{fig:B_output}.
The achieved ROC curve is shown in \cref{fig:B_ROC} with a ROC AUC of $0.739$ on the test data and $0.742$ on the training data.
Therefore, a clear separation of the $B$ mesons is achieved on simulation.
However, any algorithm that has been developed with simulated samples has to be tested on data that originates from actual measurements. 
In the following section, the developed classifier is tested on LHCb data of the decays $B_d \text{ or } B_s \rightarrow J/\psi \, K^0_\text{S}$.

\begin{figure}
    \centering
    \begin{subfigure}{0.5\textwidth}
        \centering
        \includegraphics[width=\textwidth]{images/B_output.pdf}
        \caption{distribution of $\text{Prob}_{B_s}$}
        \label{fig:B_output}
    \end{subfigure}%
    \begin{subfigure}{0.5\textwidth}
        \centering
        \includegraphics[width=\textwidth]{images/B_ROC.pdf}
        \caption{ROC curve}
        \label{fig:B_ROC}
    \end{subfigure}%
    \caption{The left figure shows the distribution of the DeepSet output split by the true labels of the simulation. The right figure shows the ROC curve of the DeepSet output. Both figures show the DeepSet prediction for the test data and the training data.}
    \label{fig:B_eval}
\end{figure}

