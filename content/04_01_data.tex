\section{Evaluation of the simulated LHCb data}

Training supervised machine learning classification algorithms requires data in which the true classes are known.
Therefore, the data used for training in this thesis originates from Monte Carlo simulations of the LHCb detector. %TODO: incorrect argument

Since both the $B_d$ and $B_s$ mesons are needed for the training of a $B_d$ and $B_s$ classifier, data of two different signal decay channels can be combined in this thesis ($B_d \rightarrow J/\psi \, K^*$ and $B_s \rightarrow D^+_s \, \pi^-$). 
Multiple datasets are available for both signal decays.
It is important to choose datasets that are compatible to each other so that the trained models do not learn artificial differences, that are not found in real detector data.
Therefore, the available datasets are compared by plotting histograms of each feature.
On the basis of this comparison, multiple sources of differences between the datasets are found.
The differences are eliminated by choosing only datasets that originate from simulations of the same year (2016) and the same simulation software version.
But because both signal decay channels are selected with different selection criteria, more differences are found.
Some features have visible differences that cannot be attributed to obvious differences in the kinematics. %TODO: is this right?
These features are excluded from the analysis of this thesis.




