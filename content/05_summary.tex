\chapter{Conclusion}

In this thesis, a classification algorithm to distinguish $B_d$ from $B_s$ mesons in $pp$-collisions is developed, trained and evaluated on LHCb data.
This algorithm is intended to be independent of the signal $B$ decay and uses only data of the associated event.
For the training of supervised machine learning algorithms, simulated data of the decays $B_d \rightarrow J/\psi \, K^*$ and $B_s \rightarrow D^+_s \, \pi^-$ detected in a simulated LHCb detector of the year 2016 are used.

The developed algorithm first uses a Boosted Decision Tree (BDT) classifier to identify \enquote{same side} (SS) fragmentation tracks.
The SS fragmentation particles are those particles that are related to the $d\bar{d}$-pair or $s\bar{s}$-pair that is needed to produce the signal $B$, and they therefore contain relevant information for the distinction between $B_d$ and $B_s$ mesons.
The other tracks in the data are tracks of the \enquote{opposite side} (OS) and tracks of particles unrelated to the signal $B$.
The OS contains particles related to the quark in the $b\bar{b}$-pair that is not in the signal $B$.
The output of this BDT is then used in a DeepSet classifier for the signal $B$.

Before training the final BDT, the available feature set is reduced.
For this, a BDT using all available features is trained and features are selected based on the gain and permutation importance of the ROC AUC. The resulting 21 features, listed in \autoref{tab:SS_features}, are used by the final BDT.
After training the final BDT, the achieved separation between SS tracks and other tracks is evaluated using a ROC curve.
The ROC curve and the distribution of the BDT output are shown in \autoref{fig:SS_eval}. A ROC AUC of $0.763$ is achieved on test data.
Although the achieved separation is not perfect, the hope is that this BDT output can help the DeepSet to form a decision.

For the feature selection of the DeepSet, the permutation importances of the accuracy and the ROC AUC are used.
This leads to the 23 features listed in \autoref{tab:B_features} that are used by the final DeepSet for the signal $B$ classification.
An indication that the BDT output ($\text{Prob}_\text{SS}$) has an impact on the DeepSet performance can be seen in \autoref{fig:B_importances}.
This however has to be interpreted with care, because feature importances can only be seen as estimates.
The trained DeepSet achieves a ROC AUC of $0.739$ on test data.
The ROC curve and the DeepSet output distribution are shown in \autoref{fig:B_eval}.

Since the given classification problem has a high complexity, the achieved separation of $B_d$ and $B_s$ meson events would potentially be useful for other studies involving uncharged $B$ mesons.
However, any algorithm that has been developed using simulated data has to be tested on \enquote{real} data. 
Also, no comparable study is found on the internet.
Therefore, the developed algorithm of this thesis is tested on LHCb data of the decays $B_d \text{ or } B_s \rightarrow J/\psi \, K^0_\text{S}$.

Compared to the training data, this LHCb data contains background events that do not actually involve $B_d$ or $B_s$ mesons.
These background events cannot be identified by the developed algorithm and must be reduced before evaluating the $B$ classification.
This background reduction is done using a combination of manual cuts and a BDT that is trained on simulated data of the decay $B_d \rightarrow J/\psi \, K^0_\text{S}$.
This BDT achieves a ROC AUC of $0.989$ on test data.
The ROC curve achieved by the BDT and the total background reduction is shown in \autoref{fig:BKG_eval}.

The developed algorithm of this thesis is then applied to the background reduced data.
To evaluate the achieved separation, $B$ mass distributions with various selection criteria based on the DeepSet output ($\text{Prob}_{B_s}$) are used to fit a function.
An example of such a fit is shown in \autoref{fig:fit_example}.
The fit components that represent the combinatorial background, the $B_d$ peak and the $B_s$ peak are then integrated numerically.
This results in estimated counts of events attributed to the $B_d$ and $B_s$ mesons remaining after each selection.

The resulting ratios of the number of $B_s$ meson events over the number of $B_d$ meson events are shown in \autoref{fig:data_ratio}.
Here, selection criteria of the form $\text{Prob}_{B_s} \geq x$ show an upward trend with increasing $x$ values.
This could be an indication of some level of achieved separation between $B_s$ and $B_d$ mesons.
However, selection criteria of the form $\text{Prob}_{B_s} \leq x$ show an increase of this ratio for lower $x$ values.
It is unclear whether these results actually show an achieved separation or result from uncertainties of the fits.
All calculated ratios are compatible with the expected ratio for no achieved separation.

Another method to test the achieved separation is to plot the efficiencies of both the $B_d$ and $B_s$ mesons similar to a ROC curve.
This is shown in \autoref{fig:data_roc}.
Here, no clear signs of any separation are shown.

All in all, the performance of the developed algorithm on simulated data can not be reproduced on \enquote{real} data.
It is unclear why this drastic performance loss occurs and whether it is possible to achieve a separation between $B_d$ and $B_s$ mesons using this algorithm.
Due to time constraints, further investigation into the reasons of this performance loss is outside the scope of this thesis.
However, the achieved separation on simulated data is an indication that the strategy to use a DeepSet to distinguish $B_d$ from $B_s$ mesons based on data of the associated event can succeed.
